\subsubsection*{Гексагон}
\begin{figure}[H]
  \centering
  \includegraphics*[width=0.25\textwidth]{./img/hexagon.pdf}
\end{figure}

\begin{itemize}
  \item Внутренний радиус = 6 км.
  \item Длина ребра = 7 км.
  \item Площадь = 124 $\text{км}^2$
\end{itemize}

\subsubsection*{Смена}
Смена - базовая единица для отслеживания времени. Смена равна 4-м (четырем) часам.

\subsubsection*{Скорость и дистанция}

\begin{mydescription}{Вынежденный марш}
  \item[Спешка] В день персонажи могут спешить не более часа. При спешке более одного часа персонажи устают,
  что даёт им штраф -1 на весь физический блок. Штраф увеличивается с каждым часом в два раза. Усталость можно снять
  отдыхом в количестве часов проведнном в спешке.
  \item[Марш] В день персонажи могут передвигать пешой скоростью 8 часов.
  \item[Форсированный марш] За каждый час сверх 8 часов марша персонажи получают штраф -1 ко всем характеристикам.
\end{mydescription}

\subsubsection*{Типы передвижения}
\begin{mydescription}{Собирательство}
  \item[Спокойное] Нет модификаторов.
  \item[Спешка] Персонажи передвигаются с большей скоростью.
  \begin{itemize}
    \item Сложность навигации +2
    \item Невозможно использовать скрытность
  \end{itemize}
  \item[Осторожное] Скорость передвижения снижена на две трети (2/3).
  \begin{itemize}
    \item Бонус +1 к Чувству Города.
    \item Можно сделать проверку Скрытности, чтобы оставаться незамеченными.
    \item Шанс на боевой энкаунтер снижена в половину. (Если он выпал, то шанс 50\%, что его не произошло)
  \end{itemize}
  \item[Исследование] Скорость передвижения снижена в половину.
  \begin{itemize}
    \item Шанс события увеличен в два раза.
    \item Невозможно использовать скрытность.
  \end{itemize}
  \item[Собирательство] Скорость передвижения снижена в половину. 
  \begin{itemize}
    \item Невозможно использовать скрытность.
    \item Бросок поиска против Сложности собирательства местности для поиска еды на целый день.
  \end{itemize}
\end{mydescription}

\subsubsection*{Местность}

Разные типы местности по разному влияют на скорость передвижения персонажей игроков

\begin{mydescription}{Бездорожье}
  \item[Проспект] Прямая, уложенная дорога.
  \item[Дорога] Менее удобные пути чем проспект, но тоже неплохо
  \item[Тропа] Неиспользующиеся пути, по которым трудно передвигаться. Ранее неизведанные тропы 
  \item[Бездорожье] Область без каких либо дорог или троп. +2 к сложности навигации.
\end{mydescription}

\begin{table}[H]
  \centering
  \def\arraystretch{1.5}
  \begin{tabular}{*6{c}}
    \toprule
    Местность & Проспект & Дорога/Тропы & Бездорожье & \makecell{Сложность \\ навигации} & \makecell{Сложность \\ собиательства} \\
    \midrule
    \multicolumn{1}{l}{Город}       & $\times 1$           & $\times 1$           & $\times \frac{3}{4}$ & 2 & 2 \\
    \multicolumn{1}{l}{Руины}       & $\times 1$           & $\times \frac{1}{2}$ & $\times \frac{1}{2}$ & 3 & 4 \\
    \multicolumn{1}{l}{Грибница}    & $\times \frac{3}{4}$ & $\times \frac{1}{2}$ & $\times \frac{1}{2}$ & 4 & 1 \\
    \multicolumn{1}{l}{Площадь}     & $\times 1$           & $\times 1$           & $\times \frac{3}{4}$ & 1 & 2 \\
    \multicolumn{1}{l}{Река}        & $\times 1$           & $\times 1$           & $\times \frac{3}{4}$ & 1 & 4 \\
    \multicolumn{1}{l}{Затопление}  & nil                  & nil                  & $\times \frac{1}{2}$ & 3 & 4 \\
    \bottomrule
  \end{tabular}
\end{table}

\subsubsection*{Источники освящения}

На четвертом ярусе неестественно темно. При отсутсвии источников освещения персонажи игроков имееют сложность навигации +2.

\subsubsection*{Отслеживание передвижения}

В качестве базовой скорости передвижения будет взято 3 км/ч. При такой скорости
персонажи игроков идут пешим шагом и смотрят под ноги.

\begin{table}[H]
  \centering
  \begin{tabular}{*2{c}}
    \toprule
    Время & \makecell{Пройденное \\ расстояние} \\
    \midrule
    1 час & 3 км\\
    4 часа & 12 км\\
    8 часа & 24 км\\
    \bottomrule
  \end{tabular}
\end{table}
