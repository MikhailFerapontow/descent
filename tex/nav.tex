\subsubsection*{Навигация по компасу}
Персонажи, которые хотят двигаться в определенном направлении должны сделать
бросок \textbf{Чувства Города} один раз за смену, чтобы не оказаться потерянными.
Персонаж со \textbf{знаниями в Инфраструктурах} не менее 4 получает бонус синергии +2 к броску навыка.

\textbf{Персонажи потерялись:} Если персонажи провалили проверку Чувства Города, они становятся
потерянными и направляются не туда куда хотели. Их отклонение от маршрута определяется броском 1d10 по
картинке ниже.

\begin{figure}[H]
  \centering
  \includegraphics*[width=0.35\textwidth]{./img/hex.pdf}
\end{figure}

Если персонажи, которые потерялись, провалят еще один бросок Чувства города их отклонение может лишь расшириться, но не сузиться.

\subsubsection*{Потерявшиеся персонажи}

\textbf{Понимание что персонажи потерялись.} Один раз за смену, потерянные персонажи могут совершить 
проверку чувства города против сложности навигации в данной местности, чтобы понять что они сбились с маршрута.

Персонажи которые встретили четкий ориентир или новую местность могут сделать дополнительный бросок
Чувства Города, чтобы понять что они сбились с маршрута. Иногда это очевидно, что бросок не нужен.

\textbf{Вернуться по своим следам.}

\textbf{Проложить новый маршрут.}

\textbf{Конфликтная ситуация.}

\subsubsection*{Поиск локации}
\textbf{Локация в поле зрения.}

\textbf{Знакомая локация.}

\textbf{Незнакомая локация.}




