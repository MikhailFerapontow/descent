\subsubsection*{Навигация}
Персонажи, которые хотят двигаться в определенном направлении должны сделать
бросок \textbf{Чувства Города} или \textbf{Поиск} один раз за смену, чтобы не оказаться потерянными.
Персонаж со \textbf{знаниями в Инфраструктурах} не менее 4 получает бонус синергии +2 к броску навыка.

\textbf{Персонажи потерялись:} Если персонажи провалили проверку Чувства Города, они становятся
потерянными и направляются не туда куда хотели. Их отклонение от маршрута определяется броском 1d10 по
картинке ниже.

\begin{figure}[H]
  \centering
  \includegraphics*[width=0.35\textwidth]{./img/hex.pdf}
\end{figure}

Если персонажи, которые потерялись, провалят еще один бросок Чувства города их отклонение может лишь расшириться, но не сузиться.

\subsubsection*{Потерявшиеся персонажи}

\textbf{Понимание что персонажи потерялись.} Один раз за смену, потерянные персонажи могут совершить 
проверку чувства города против сложности навигации в данной местности, чтобы понять что они сбились с маршрута.

Персонажи которые встретили четкий ориентир или новую местность могут сделать дополнительный бросок
Чувства Города, чтобы понять что они сбились с маршрута. Иногда это очевидно, что бросок не нужен.

\textbf{Переориентация.} Персонажи, которые поняли что потерялись, имеют несколько опций для возвращения себя на правильный маршрут:

\textit{Вернуться по своим следам.} Персонажи могут вернуться по своим следам, но они все еще должны понять
в какой момент они потерялись. Для этого они могут раз за смену совершить бросок Чувства Города/Поиска против
сложности навигации местности. Если проверка успешна, то персонажи точно находят точку, где потерялись. Если
проверка неуспешна, то есть 75\% шанс, что персонажи придут к ложному заключению.

\textit{Проложить новый маршрут.} Потерянные персонажи могут проложить новый точный маршрут.
Для этого им нужно сделать проверку против сложности навигации в этой местности +1. Если
персонажи провалили проверку, то они немедленно становятся потерянными.

\textbf{Конфликтная ситуация.}
Если несколько персонажей одновременно пытаются определить правильное направление, то должны
сделать проверку Чувства Города/Поиска секретно. Персонажи, которые преуспели в проверке, получают правильное
направление движения, остальные получают случаное направление, которое они считают правильным.

\subsubsection*{Поиск локации}
\textbf{Локация в поле зрения.}
а это точно будет работать в столице?

\textbf{Локация на дороге.} Если локация находится на проспекте, реке или знакомой дороге/тропе, то персонажи
автоматически её находят, если она конечно не спрятана.

\textbf{Незнакомая локация.}
Персонажи специально тратят поиск в конкретной локации в режиме исследования.
Их действия не приближают их к выходу из хексагона. В это время мастер делает нужные проверки
на случайное событие.



