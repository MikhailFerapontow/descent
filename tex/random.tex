\subsection*{Вероятность события}

Делайте проверку один раз за смену, чтобы определить произошло событие или нет.

\begin{table}[H]
  \centering
  \begin{tabular}{*3{l}}
    \toprule
    Проверка & Смена & День \\
    \midrule
    1 в 1d6 & 16\% & 66\% \\
    2 в 1d6 & 33\% & 91\% \\
    1 в 1d8 & 13\% & 57\% \\
    2 в 1d8 & 25\% & 82\% \\
    1 в 1d10 & 10\% & 46\% \\
    2 в 1d10 & 20\% & 73\% \\
    1 в 1d20 & 5\% & 26\% \\
    \bottomrule
  \end{tabular}
\end{table}

\subsubsection*{События во время исследования}

Данные события случаются только во время смен, во время которых персонажи путешествовали
или исследовали территорию. Они не могут произойти во время смен, когда персонажи оставались на месте
или отдыхали.

\textbf{Локация:} Это событие указывает, что персонажи наткнулись на ключевое место в гексагоне.
Большинство гексагонов имеют одно ключевое место.

\textbf{\% Логово:} Проценты, в таблице ниже, определеют встретили персонажа существо в его логове.
Данная информация добавляется на гексагон, тем самым заполняя их изначальную пустоту.

\textbf{\% Следы:} Проценты, в таблице ниже, определеют нашли ли персонажи лишь следы проходившего здесь существа
(но не само существо). Персонажи могут сделать проверку Поиска/Чувства Города, чтобы заметить следы. Они обычно
1d10 дневной давности. Мастер может определить куда они ведут, обычно в их логово.

\textit{Подсказка:} Проверьте является ли событие следами. Если нет, то является ли это логовом.
Если нет, то это случайная встреча.

\subsection*{Пример таблицы случаных событий}

Нахождение локации: 1 в 1d6

Вероятность события: 1 в 1d8

\begin{table}[H]
  \centering
  \begin{tabular}{*4{l}}
    \toprule
    1d20 & Проверка & \% логово & \% следы \\
    \midrule
    1-3   & Беженцы        & 30\% & 30\% \\
    4-5   & Исполнители    & 30\% & 30\% \\
    6     & Серпари        & 30\% & 30\% \\
    7-9   & Бандиты        & 30\% & 30\% \\
    10-12 & Повстанцы      & 30\% & 30\% \\
    13    & ГИЗ            & 20\% & 30\% \\
    14    & Трубачи        & 30\% & 30\% \\
    15    & Культисты      & 30\% & 30\% \\
    16    & Контрабандисты & 30\% & 30\% \\
    17-18 & Наркоторговцы  & 30\% & 30\% \\
    19    & Коллекторы     & 30\% & 30\% \\
    20    & брось 2 раза   & nil & nil \\
    \bottomrule
  \end{tabular}
\end{table}

\textit{Заметка:} В таблице можно указаывать места возможного нахождения логова существ,
если мастер его уже нанес на карту.

\subsubsection*{Отношение к персонажам}

Если вам тяжело определить реакцию случайно встреченнных существ
на персонажей игроков, то вы можете кинуть 1d12 кубик по следующей таблице:

\begin{table}[H]
  \centering
  \begin{tabular}{*4{l}}
    \toprule
    1d12 & Реакция \\
    \midrule
    1-3   & Немедленная атака\\
    4-5   & Враждебная\\
    6-8   & Осторожная\\
    9-10  & Нейтральная\\
    11-12 & Дружелюбная\\
    \bottomrule
  \end{tabular}
\end{table}